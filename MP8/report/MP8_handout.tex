\documentclass[12pt]{extarticle}
\usepackage{amsmath, amsthm, latexsym, tikz, graphicx, listings, microtype, mathtools, soul, color, fancyhdr}
\usepackage[margin=1in]{geometry}
\usepackage[utf8]{inputenc}

\newenvironment{myindentpar}[1]%
 {\begin{list}{}%
         {\setlength{\leftmargin}{#1}}%
         \item[]%
 }
 {\end{list}}
 
\DeclarePairedDelimiter\abs{\lvert}{\rvert}%
\DeclarePairedDelimiter\norm{\lVert}{\rVert}%

% Swap the definition of \abs* and \norm*, so that \abs
% and \norm resizes the size of the brackets, and the 
% starred version does not.
\makeatletter
\let\oldabs\abs
\def\abs{\@ifstar{\oldabs}{\oldabs*}}
%
\let\oldnorm\norm
\def\norm{\@ifstar{\oldnorm}{\oldnorm*}}
\makeatother

\definecolor{lightgray}{gray}{0.65}
\definecolor{pinegreen}{RGB}{1, 171, 161}
\definecolor{lightblue}{RGB}{135, 206, 250}
\definecolor{dkgreen}{rgb}{0,0.6,0}
\definecolor{gray}{rgb}{0.5,0.5,0.5}
\definecolor{mauve}{rgb}{0.58,0,0.82}
\definecolor{darkblue}{rgb}{0.0,0.0,0.6}
\definecolor{cyan}{rgb}{0.0,0.6,0.6}

\newcommand*{\Value}{\frac{1}{2}x^2}%
\newcommand{\hlc}[2][yellow]{ {\sethlcolor{#1} \hl{#2}} }


% /*--------------------------------------------------------------*/
%   Changing the values here sets the due date for the assignment!
% /*--------------------------------------------------------------*/
\newcommand{\duedate}{XX/XX/XX }
\newcommand{\semester}{SEMESTER}

\lstset{frame=tb,
  language=C++,
  breaklines=true,
  showstringspaces=false,
  columns=flexible,
  numbers=none,
  tabsize=3,
  escapeinside={(*@}{@*)}
  %,
  %commentstyle=\color{dkgreen},
  %stringstyle=\color{mauve}
}
\pagestyle{fancy}
\fancyhf{}
\renewcommand{\headrulewidth}{0pt}
\lhead{\color{lightgray} CSCE-313}
\rhead{\color{lightgray} \semester}
\rfoot{\thepage}
\pagenumbering{arabic}

\definecolor{codegray}{gray}{0.9}
\newcommand{\code}[1]{\colorbox{codegray}{\texttt{#1}}}

\begin{document}
\begin{center}
    \underline{{\large Machine Problem 8: Synchronization \  }(Due: \duedate)}  \\
\end{center}

\ \\
{\large \underline{Introduction}:}

\begin{myindentpar}{5mm}
    
	\noindent
	You may have noticed that the client from MP7 had several limitations:
	\begin{enumerate}
		\setlength\itemsep{-0.1em}	
	\item The request buffer had to be populated all at once, before the worker threads begin, which is not adaptable to real-time.
	\item The request buffer was susceptible to grow to infinity, which can cause problems with memory.
	\item The worker threads are in charge of assembling the final representation of the data, which is an example of poor modularity: a histogram of bin size 10 isn’t always the best representation of the data, and changing it shouldn’t require changing how responses are received.  
	\end{enumerate}
	Addressing these problems requires more advanced synchronization than we were ready to handle during MP7, but by this point we should be ready to fix things up a bit.
    
\end{myindentpar}

\ \\
{\large \underline{Background}:}

\begin{myindentpar}{5mm}

    \noindent
    \ \\
    \textbf{A Classic Synchronization Problem}
    
    \vspace{-3mm}
    \ \\
    To address the first limitation, one could simply allow the request threads to run in parallel with the worker threads. The only new challenge (and one that you will have to address as part of this assignment) would seem to be ensuring proper termination of the worker threads.
    
    \ \\
	But if we look a little closer, we see that the problem is much more complicated than that. What if, due to the unpredictability of the thread scheduler or the data server, the worker threads processed all the requests in the request buffer before the request threads had finished? Maybe the worker threads wouldn’t terminate, but if we stick with the MP7 code then they would try to draw from an empty data structure. This highlights one of the glaring problems with using plain-old STL (or otherwise conventional) data structures in a threaded environment, even if they have mutexes wrapped around them like the SafeBuffer class did: underflow is possible. Clearly the worker threads need to wait for requests to be added to the request buffer, but what’s the best way to do that?
	
	\ \\
	This is closely tied to the second limitation from the introduction: the request buffer is susceptible to grow to infinity (or in our case, n), in other words it allows overflow. However, placing an artificial ceiling on the number of  requests is not a realistic solution. The request threads need wait for worker threads to deplete the buffer to a certain point before pushing new requests to it, which brings us back to our earlier question: what’s the best way to do that?
	
	\ \\
	The synchronization concern in MP7 was concurrent modification, or interleaving. The new synchronization concerns of MP8 combine to form one of the classic synchronization problems that will be/has been discussed in lecture: the producer-consumer problem. It’s a fairly common programming problem, and one which has a bounty of real-life applications. There are some naïve solutions which may be/may have been discussed in lecture, but there’s not room to discuss them here.
	
	\ \\
    \textbf{Let's Try a New Data Structure!}
    
    \ \\
    All the problems discussed so far are problems with the data structure used for the request buffer, so the solution could be a new data structure. If that were the case, the new data structure would need to:
    \begin{itemize}
    	\setlength\itemsep{-0.1em}
    	\item Prevent underflow
    	\item Prevent overflow
    	\item Prevent concurrent modification
    \end{itemize}

	Because it is “bounded” on both sides, we call this data structure a bounded buffer. Now we must ask, how can we build a bounded buffer? 
	
	\ \\
    \textbf{Another Synchronization Primitive: Semaphore}
    
    \ \\
	We can build one using a new (to us) synchronization primitive, called a semaphore. A semaphore is a variable that keeps track of how many units of a given resource are available. In our case, the resource is the number of available spaces for requests in the bounded buffer. While the mutex primitive provides operations that lock and unlock the mutex, the semaphore provides operations that decrement and increment the record of the amount of resource available. When a thread or process needs the resource it decrements the semaphore, and if the semaphore’s count passes below 0 then the process enters a wait queue and sleeps. When more of the resource becomes available the semaphore count is incremented and, if the count passes from -1 to 0, then the scheduler picks a process from the semaphore wait queue and wakes it up. Traditionally the semaphore’s decrement/wait and increment/wake up operations are called P() and V() respectively. 

	\ \\
    \textbf{Putting it all together}
    
    \ \\
	Now we can explain how using a bounded buffer to hold requests prevents underflow and overflow. It uses two semaphores, which we call full and empty, which  represent (respectively) the number of requests in the bounded buffer and the number of free spaces in it. When a thread needs to remove a request from the bounded buffer for processing, internally the bounded buffer calls full.P(), then removes the next request, then calls empty.V(). The opposite occurs when a thread needs to add a request to the bounded buffer: internally the bounded buffer calls empty.P(), adds the request, then calls full.V(). When the bounded buffer is constructed the counter for the empty semaphore is given an initial value that represents the maximum desired size of the bounded buffer, while the counter for the full semaphore is given the initial value of 0. 
	
	\ \\
	Take a moment to think about how this works. When the full semaphore reaches zero there are no more requests left in the buffer, so processes that decrement it past zero wait instead of attempting to draw from an empty queue. When the empty semaphore reaches zero the buffer has reached the maximum allowed size, so processes that decrement it past zero will wait instead of growing the buffer to infinity. 
	
	\ \\
	Because multiple processes/threads might “obtain a semaphore” (successfully call P() and be granted access to the buffer) at the same time it is still necessary for operations on the underlying data structure to be protected by a mutex to prevent concurrent modification. This mutex should be contained in the bounded buffer just like the full and empty semaphores.

	\ \\
    \textbf{Addressing the last limitation}
    
    \ \\
	The bounded buffer data structure suffices to address the first two limitations stated in the intro section. So what about the third? The solution is fairly simple: instead of having the worker threads directly modify the frequency count vectors for the three patients, make three response buffers (one per patient) and have the worker threads just sort responses into the correct one. Then, have three statistics threads (again, one per patient) remove responses from the response buffers and process them however. For this assignment the result will still be a histogram with bin size 10, but theoretically it could be anything appropriate.
	
	\ \\
	This solution introduces a separate producer-consumer problem, where instead of the request threads being the producers and the worker threads being the consumers we have the worker threads being the producers and the statistics threads being the consumers. Since we already have the bounded buffer data structure available to use, we can also use it for the patient response buffers and the problem is solved. There you go.	
	
\end{myindentpar}

\ \\
{\large \underline{Assignment}:}

\begin{myindentpar}{5mm}

	\ \\
    \textbf{Code}	

	\ \\
	You are given the same files as in MP7 (dataserver.cpp, reqchannel.cpp/.h, functioning makefile), except that the client is called client\_MP8.cpp and has been changed some. You will have to modify it so that it has the same functionality as the completed client\_MP7.cpp, but the request threads, worker threads, and statistics threads will all run in parallel. To make this requirement a bit clearer, here are some rules of thumb to follow:
	\ \\
	Your code may NOT (i.e. points will be deducted if it does):
	\begin{itemize}
		\setlength\itemsep{-0.1em}
		\item Wait for any thread to finish before all threads have been started
		\item Process requests or responses in a non-FIFO order 
			\begin{itemize}
			\setlength\itemsep{-0.1em}
			\item This means that requests and responses must be removed from their respective buffers in the order in which they were added to those same buffers.
			\end{itemize}
		\item Push data requests to the request buffer outside the request thread function 
			\begin{itemize}
			\setlength\itemsep{-0.1em}
			\item Note that any other kind of request may be pushed or sent inside main, but data requests can only be handled by the thread functions
			\end{itemize}
		\item Modify the frequency count vectors outside the statistics thread function (it is fine to initialize them inside main)
	\end{itemize}

	\ \\
	In addition to the command-line arguments from MP7 (“n” for number of requests per patient and “w” for number of worker threads), your program must take an additional command-line argument: “b,” for the maximum size of the request buffer. The patient response buffers can be of an arbitrary fixed size.
On that note, to make any of this possible you will need to implement the BoundedBuffer and Semaphore classes. They must at least have their own dedicated .h files, though you can also have corresponding .cpp files if you find it helpful. However, in any case you will lose points if these classes are defined in client\_MP8.cpp: they must have their own files. Note that the BoundedBuffer class will have to use the Semaphore class, but it will not be necessary to use the Semaphore class anywhere else.

	\ \\
	The Semaphore class must support at least the following operations:
	\begin{itemize}
		\setlength\itemsep{-0.1em}
		\item Construction: initialize mutex, counter, and wait queue members. The constructor should take a single int argument for the initial counter value.
		\item Destruction: Properly clean up all non-primitive members (i.e. calling pthread\_mutex\_destroy, etc.), and return all heap memory to the heap if any was used.
		\item P: locks a mutex, then decrements the counter. If counter goes below zero, release mutex and enter wait queue. Else release mutex.
		\item V: locks a mutex, then increments the counter. If previous counter value was negative and current value is non-negative, signal the wait queue. Then release mutex.
	\end{itemize}

	\ \\
	You may be wondering how one is supposed to do all the wait queue operations, since the language on that point has been vague so far. We recommend letting the API do the wait queue operations for you: man 3 pthread\_cond\_init, man 3 pthread\_cond\_signal, man 3 pthread\_cond\_wait, man 3 pthread\_cond\_destroy. The wait queue member of the Semaphore class can be a single pthread\_cond\_t variable.
	
	\ \\
	The BoundedBuffer class must support at least the following operations:
	\begin{itemize}
		\setlength\itemsep{-0.1em}
		\item Construction: initialize access mutex, full and empty semaphores, and (if necessary) the underlying data structure
		\item Destruction: properly clean up all non-primitive members, return all heap memory to the heap if any was used
		\item Element addition: add element to underlying data structure, with proper use of semaphores and mutexes. Ensure FIFO order.
		\item Element removal: remove element from underlying data structure and return it, with proper use of semaphores and mutexes. Ensure FIFO order.
	\end{itemize}

	\ \\
	Since requests must be processed in a FIFO order, we recommend you use something like std::queue or std::list for the underlying data structure. Adding and removing elements from std::vector at the front is very expensive and will impact the performance of your program.

	\ \\
    \textbf{Bonus}	

	\ \\
	You have the opportunity to gain bonus credit for this machine problem. To gain this bonus credit, you must implement a real-time histogram display for the requests being processed. 
	
	\ \\
	You may remember the final display from MP7. The make\_histogram\_table function is used to format and output the frequency counts for the requests processed for each patient. This function takes as arguments the 3 patient names and pointers to the 3 *frequency\_count vectors. 
	
	\ \\	
	Write a signal-handler function that clears the terminal window (system(”clear) is an easy way to do this) and then displays the output of make\_histogram\_table. You will need to make sure that this is threadsafe by synchronizing on the same mutexes that the worker threads use to modify the *frequency\_count vectors. 
	
	\ \\
	In main, register your signal-handler function as the handler for SIGALRM (man 2 sigaction). Then, set up a timer to raise SIGALRM at 2-second intervals (man 2 timer create, man 2 timer settime), so that your handler is invoked and displays the current patient response totals and frequency counts approximately every 2 seconds. To do this, you will need to make sure that your signal handler is given all the necessary parameters when it catches a signal (man 7 sigevent). When all requests have been processed, stop the timer 
(man 2 timer delete). 

	\ \\
	If you have succeeded, the result should look like a histogram table that stays in one place in the terminal while its component values are updated to reflect the execution of the underlying program. 
Is there some purpose for this bonus portion other than making your program output look nice? Yes, there is. Firstly it gives you the opportunity to use signals and signal handlers, which are important both in this course and generally in computer science. Secondly it provides a snapshot of the run-time state of your program, which can be useful for debugging (i.e. if you can tell that the histogram is updating but the values aren’t changing, you know there’s a problem to be diagnosed). 

	\ \\
	As with the other parts of the assignment you will lose points for using global variables. However, using global variables for the bonus will result in points being deducted only from the bonus portion of the assignment, not from the main assignment, i.e. you cannot lose points by attempting to earn the bonus credit. Finally, please use the make\_histogram\_table function to format your output. Programming assignments are much easier to grade when the output is standardized. 

	\ \\
    \textbf{Report}	

	\begin{enumerate}
		\setlength\itemsep{-0.1em}
		\item Present a brief performance evaluation of your code. If there is a difference in performance from MP7, attempt to explain it. If the performance appears to have decreased, can it be justified as a necessary trade-off?
		\item Make two graphs for the performance of your client program with varying numbers of worker threads and varying size of request buffer (i.e. different values of “w” and “b”) for n = 10000. Discuss how performance changes (or fails to change) with each of them, and offer explanations for both.
			\begin{itemize}			
			\setlength\itemsep{-0.1em}
				\item Include b = 1 is among your data points.
				\item Please don’t copy-paste from your MP7 report, even if some of the answers you come up with are similar for MP8.
			\end{itemize}
		\item Describe the platform that your data was gathered on and the operating system it was running. A simple description like “a Raspberry PI model B running Raspbian OS,” or “the CSE Linux server,” is sufficient. (Think of this as free points)
	\end{enumerate}

	\ \\
    \textbf{What to Turn In}	

	\begin{itemize}
		\setlength\itemsep{-0.1em}
		\item All original .cpp/.h files, and the makefile, with whatever changes you made to complete the assignment
		\item Any additional files you used to compile/run your program
		\item Completed report
	\end{itemize}

	\ \\
    \textbf{Rubric}	

	\begin{enumerate}
		\setlength\itemsep{-0.1em}
		\item BoundedBuffer class (20 pts)
		\item Semaphore class (20 pts)
		\item Having BoundedBuffer and Semaphore as separate classes, not in the main. This should be reflected in the makefile. If you have these in the main, you will lose 10 points
		\item Correct values of the frequency counts (20 pts) 
			\begin{itemize}
			\setlength\itemsep{-0.1em}
			\item This is important since it indicates whether your bounded buffer and semaphore implementations are correct
			\end{itemize}
		\item Not having global variables (10 pts)
		\item Bonus: using timers to display the counts (20 pts)
			\begin{itemize}
			\setlength\itemsep{-0.1em}
			\item If you use global variables, you only get 10 bonus pts (again, no points will be deducted from the main part of the assignment)
			\item If your implementation uses a separate thread instead of a signal handler, you only get 5 bonus pts
			\end{itemize}
		\item Report (20 pts)
			\begin{itemize}
			\setlength\itemsep{-0.1em}
			\item Should show plots  of runtime under  varying n, b, w. Specially we want to see variations in b and w after setting n=10K at least.
			\end{itemize}
	\end{enumerate}

\end{myindentpar}

\end{document}